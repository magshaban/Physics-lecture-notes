\documentclass[english,seminar,headertitle]{lecture}

\newcommand{\lag}{\mathscr{L}}
\newcommand{\ham}{\mathscr{H}}
\newcommand{\kam}{\mathscr{K}}

\title{Advanced classical mechanics}
\subtitle{Poisson brackets}
\shorttitle{}
\ccode{16MSPAH101}
\subject{Classical Mechanics}
\speaker{V.H. Belvadi}
\spemail{vh@belvadi.com}
\author{}
\email{}
\flag{}
\season{Autumn 2017}
\date{}{}{}
\dateend{}{}{}
\conference{}
\place{St Philomena's College}
\attn{}
\morelink{vhbelvadi.com/teaching}

\begin{document}
	
\runin{One of the} most important tools at our disposal in Hamiltonian mechanics are Poisson brackets. These help us determine if transformations are canonical and, on a deeper level, they act as a sort of bridge between classical and quantum mechanics.

\section{Poisson brackets}
\subsection{Definition}
\margintext{The name Poisson is pronounced like pwa-son and nasalised, not like poison. We are talking of the eighteenth century French physicist Sim\'{e}on-Denis Poisson.}
Consider two functions $f$ and $g$ of the variables $p$ and $q$ our usual generalised momenta and co\"{o}rdinates. The Poisson bracket of $f$ and $g$ is defined as
$$
[f, g]_{p,q} \equiv \partial_q f \partial_p g - \partial_p f \partial_q g
$$
and may be generalised to
\begin{equation}
	[f, g] \equiv \partial_{q_i} f \partial_{p_i} g - \partial_{p_i} f \partial_{q_i} g \label{eq:pbrackets}
\end{equation}%
with summation implied as per Einstein's convention. And this gives us two important properties of the Poisson bracket:
\begin{equation}
	[q_i,q_j] = 0 = [p_i,p_j] \qquad\textrm{and}\qquad	[q_i,p_j] = -[p_i,q_j] = \delta_{ij} \label{eq:basic-prop}
\end{equation}%
where $\delta_{ij}$ is the Kronecker delta function. But the property that makes Poisson brackets particularly useful for canonical transformations is that during a transformation Poisson brackets remain invariant:
\begin{equation}
	[f,g]_{p,q} = [f,g]_{P,Q} \label{eq:invariance}
\end{equation}

\subsection{Rules}
\margintext{If you are wondering where Poisson brackets appear in quantum mechanics, we use the operator-led definition $[f,g] = (\hat{f}\hat{g} - \hat{g}\hat{f})/i\hbar$ quite often as commutators.}
Poisson brackets, like any mathematical tool, adhere to certain rules that help us manipulate them. Think of this as analogous to the basic arithmetic operations for Poisson brackets.
\begin{equation}
\begin{array}{ll}
	[f,f] = 0 \quad & \quad [f,g] = -[g,f] \\[1em]
	\left[\alpha f + \beta g, h\right] = \alpha [f, h] + \beta [g,h] \quad & \quad [fg,h] = [f,h]g + f[g,h]
\end{array} \label{eq:rules}
\end{equation}%
The first two of these are the same as those mentioned in the previous section; and all four can be proven by simply expanding the brackets as in eq. \eqref{eq:pbrackets}.

There is a fifth, extremely important cyclic relationship known as the Jacobi identity that Poisson brackets obey:
\begin{equation}
	[f,[g,h]] + [g,[h,f]] + [h,[f,g]] = 0 \label{eq:jacobi-identity}
\end{equation}

\section{Equations of motion}

In the same vein as the Newtonian approach, the Lagrangian approach (both of which involved second order derivatives), and the Hamiltonian approach (a first order derivative) we can write the equations of motion of a body in terms of Poisson brackets.

For some $f$ we have
$$
\dot{f} = \partial_{q_i} f \dot{q_i} + \partial_{p_i} f \dot{p_i} + \partial_t f
$$
where $\dot{q}_i = \partial_{p_i} \ham$ and $\dot{p}_i = - \partial_{q_i} \ham$ giving us
$$
\dot{f} = \partial_{q_i} f \partial_{p_i} \ham - \partial_{p_i} f \partial_{q_i} \ham + \partial_t f
$$
which is nothing but
\begin{equation}
\dot{f} = [f,\ham] + \partial_t f \label{eq:dotted}
\end{equation}%
Now if $f$ is a constant in time $\dot{f} = 0$ and we have
\begin{equation}
	[f,\ham] = -\partial_t f \label{eq:motion}
\end{equation}
If, in eq. \eqref{eq:motion}, we replace $f$ by either $\dot{q}$ or $\dot{p}$ we end up with Hamilton's equation in Poisson brackets notation:
$$
\dot{q} = [q,\ham] \qquad\textrm{and}\qquad \dot{p} = [p,\ham]
$$
And if, in the same equation, we replace $f$ by $\ham$ we get $\dot{\ham} = \partial_t \ham$ which is a result we have come across in previous lectures.

\subsection{Infinitesimal transformations in space and time}

Infinitesimal transformations are transformations going from $(p_i,q_i)$ to $(P_i,Q_i)$ with only an infinitesimal change: $Q_i = q_i + \delta q_i$ and $P_i = p_i + \delta p_i$, where $\delta p$ and $\delta q$, like $p$ and $q$, is any co\"{o}rdinate change, and not just virtual displacement.

Comparing with the generating function $\chi'' = q_iP_i$ we can think of our new infinitesimal transformation as being generated by
$$
\chi'' = q_iP_i + \epsilon G(q,P,t)
$$
and, using the normal transformation rules for a $\chi''$ type generating function, i.e. $p_k = \partial_{q_k} \chi''$ and $Q_k = \partial_{P_k} \chi''$, we get
$$
p_k = P_k + \epsilon \partial_{q_k} G \qquad\textrm{and}\qquad Q_k = q_k + \epsilon \partial_{P_k} G
$$
Subsequently we have $\delta q_k = Q_k - q_k = \epsilon \partial_{P_k} G$ which can be written equivalently as $\delta q_k = \epsilon \partial_{p_k}$ since $P_k$ and $p_k$ vary only as a first order transformation. Likewise $\delta p_k = P_k - p_k = -\epsilon \partial_{q_k} G$.

In terms of the Hamiltonian this means any transformation $\xi = \eta + \delta \eta$ is a canonical transformation where, if $G \equiv \ham$, these transformations \textit{describe the growth of our generalised variables in time} e.g. $q(t), \ldots, p(t)$ going to some $q(t+\delta t),\ldots,p(t+\delta t)$.

It is worth taking a moment to discuss what these transformations will \textit{look} like. For starters going from $f(q,p)$ to $F(Q,P)$ purely spatially will not change the value of $f$ in any way, i.e. we end up with $f \equiv F$ in all cases for canonical transformations. However, as a mathematical equation $f$ will look different from $F$. This is not surprising: we have seen examples before of how the same system is described by different equations in, say, Cartesian and polar co\"{o}rdinates although the equations physically mean the same thing.

With temporal transformations, though, things are quite the opposite. Since we remain in the same co\"{o}rdinate space but move in time from $f(t)$ to $f(T)$ the form of $f$ remains unchanged while its value changes and $f(t) \neq f(T)$. Let us give this a general representation: say we start from $f(q,p)$ at $q(t),\ldots,p(t)$ and go to $f(q',p')$ at $q(t+\delta t),\ldots,p(t+\delta t)$ and represent this as $f(A)$ and $f(B)$ respectively, we define some
$$
\delta f = f(B) - f(A)
$$
and start using $f \equiv f(q,p)$ to solve this as
\begin{align}
	\delta f &= \partial_{q_i} f \delta q_i + \partial_{p_i} f \delta p_i \nonumber\\
			&= \epsilon \left( \partial_{q_i} f \partial_{p_i} G - \partial_{p_i} f \partial_{q_i} G \right) \nonumber\\
\implies \delta f &= \epsilon [f, G] \label{eq:fB-A}
\end{align}

\subsection{Transforming the Hamiltonian}

Finally, consider the transformation $\ham(A) \rightarrow \kam(B)$ of the Hamiltonian of some system related by $\kam(B) = \ham(A) + \partial_t \chi$. For an infinitesimal transformation we can consider the identity transformation $\chi'' = q_iP_i + \epsilon G(q,P,t)$ so that
$$
\kam(B) = \ham(A) + \epsilon \partial_t G
$$

The change in the Hamiltonian $\ham(A)$ to $\ham(B)$ is related to the above transformation by some
\begin{align*}
	\delta \ham &= \ham(B) - \kam(B) \\
				&= \ham(B) - \ham(A) - \epsilon \partial_t G \\
	\delta \ham &= \epsilon[\ham, G] - \partial_t G \qquad\qquad\textrm{from eq. \eqref{eq:fB-A}}
\end{align*}

From eq. \eqref{eq:dotted} we have
$$
\dot{G} = [G,\ham] + \partial_t G
$$
which can be substituted in the above result to get
\begin{equation}
	\delta \ham = - \epsilon \partial_t G
\end{equation}
which means if the generating function (on the right-hand side) is a constant of motion the Hamiltonian remains invariant and vice versa.

\vspace*{1.5cm}
{\centering
* \; * \; *

}
\end{document}
