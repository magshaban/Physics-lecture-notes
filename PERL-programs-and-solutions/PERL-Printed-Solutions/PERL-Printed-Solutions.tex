%%%%%%%%%%%%%%%%%%%%%%%%
%%%%%%%%%%%%%%%%%%%%%%%%
%%%%%%%%% YCM %%%%%%%%%%
%%% PERL SOLUTIONS %%%%%
%%%% V.H. BELVADI %%%%%%
%%%%%%%% 2020 %%%%%%%%%%
%%%%%%%%%%%%%%%%%%%%%%%%
%%%%%%%%%%%%%%%%%%%%%%%%
% Download a copy at %%%
% vhbelvadi.com/teaching
%%%%%%%%%%%%%%%%%%%%%%%%
%%%%%%%%%%%%%%%%%%%%%%%%
% Compile with XeLaTeX % <-----
%% Charter and Avenir %% <-----
%% Next fonts needed %%% <-----
%%%%%%%%%%%%%%%%%%%%%%%%
%%%%%%%%%%%%%%%%%%%%%%%%
% Alright, let’s begin %
%%%%%%%%%%%%%%%%%%%%%%%%
\documentclass[11pt,oneside]{article}
%%%%%%%%%%%%%%%%%%%%%%%%
%%%%%%%%%%%%%%% PACKAGES
%%%%%%%%%%%%%%%%%%%%%%%%
\usepackage[a4paper, margin=1in,top=1.2in,bottom=1.25in,left=1.05in]{geometry}
\usepackage{listings}
\usepackage[most]{tcolorbox}
\usepackage{xltxtra}
\usepackage[small]{titlesec}
\usepackage{hyperref}
%%%%%%%%%%%%%%%%%%%%%%%%
%%%%% REWRITING COMMANDS
%%%%%%%%%%%%%%%%%%%%%%%%
\renewcommand{\baselinestretch}{1.35}
\renewcommand{\lstlistingname}{Program}
%%%%%%%%%%%%%%%%%%%%%%%%
%%%%%%%%%%%%%%%% TCBoxes
%%%%%%%%%%%%%%%%%%%%%%%%
\DeclareTotalTCBox{\synt}{ v }{verbatim, colframe=black,left=0em,right=0em,sharpish corners,colback=black!20!white,boxrule=.2mm}{#1}
\newtcolorbox{infobox}{colback=black!10!white, colframe=black!10!white,left=0.5em,right=0.5em,before skip=.75em,middle=0em,sharpish corners,oversize}
%%%%%%%%%%%%%%%%%%%%%%%%
% SETTING UP CODE BLOCKS
%%%%%%%%%%%%%%%%%%%%%%%%
\lstset{language=C,basicstyle=\ttfamily,numbers=left,numbersep=22pt,numberstyle=\ttfamily\color{black!50!white},firstnumber=last,showspaces=false,showstringspaces=false,belowskip=1.5em,aboveskip=1.5em,breaklines=true,xleftmargin=1.55em,frame=lr,framesep=8pt,framerule=0pt,captionpos=b,escapeinside={(*@}{@*)}}
%%%%%%%%%%%%%%%%%%%%%%%%
%%%%%%%%%%%%%% TYPEFACES
%%%%%%%%%%%%%%%%%%%%%%%%
\setmainfont{Charter}
\setsansfont{Avenir Next Medium}
\setsansfont[BoldFont=Avenir Next Bold]{Avenir Next Bold}
%%%%%%%%%%%%%%%%%%%%%%%%
%% DEFINING NEW COMMANDS
%%%%%%%%%%%%%%%%%%%%%%%%
\newcommand{\info}[1]{\textbf{\scriptsize{\sffamily\addfontfeatures{LetterSpace=7} #1\\[.35em]}}}
\newcommand{\LetterSpaced}{\addfontfeatures{LetterSpace=7}}
%%%%%%%%%%%%%%%%%%%%%%%%
%% DEFINING TITLE FORMAT
%%%%%%%%%%%%%%%%%%%%%%%%
\titleformat{\section}{\small\sffamily\bfseries}{\thesection.}{1em}{\LetterSpaced\MakeUppercase }
%%%%%%%%%%%%%%%%%%%%%%%%
% LET’S BEGIN FOR REAL %
%%%%%%%%%%%%%%%%%%%%%%%%
\begin{document}
%
% TITLE
%
{\centering
\textsf{\textbf{YPH 205}}

\textsf{{\large\MakeUppercase{Solutions to prescribed PERL programs}}}

}
%
% Raw file notice
%
\vspace{0.25cm}
\noindent \small{This document was typeset using ${\LaTeX}$. If you are interested, you can download and examine the \verb+.tex+ file at \href{https://vhbelvadi.com/teaching}{\texttt{vhbelvadi.com/teaching}}, suggest edits and alternate solutions and more. Updated 09.02.2020.}
%
% EXECUTION INSTRUCTIONS
%
\vspace*{-0.75ex}
\begin{infobox}
\info{\MakeUppercase{Execution instructions for all programs}}
1. Open Terminal, Command Prompt or equivalent Command Line Interface.

2. Create a new folder with \synt{mkdir <folderName>}. \textbf{Use new folders for each program.}

3. Open the new folder with \synt{cd <folderName>} and type \synt{vi <name>.pl} to open a new file.

4. Type your program in the file. Use \synt{I} to start typing and \synt{Esc} to stop typing.

5. After typing, hit \synt{Esc} then save the program with \synt{:w} and quit with \synt{:q}, or just use \synt{:wq}.

6. Run the program with \synt{perl <name>.pl}.

\textit{In all cases above, replace} \synt{<...>} \textit{with a name of your choice (obviously). Avoid spaces in file names.}
\end{infobox}
%
% 1. Finding the roots of a quadratic equation
%
\section{Searching for a pattern in a string}
\begin{infobox}
\info{\MakeUppercase{Special instructions for this program}}
1. Create a text file \synt{vi input.txt} with three lines of data e.g.:

\qquad\qquad This is a sample

\qquad\qquad text file used

\qquad\qquad for PERL programs.

2. Create a perl file \synt{vi pattern-match.pl} and type the program below.
\end{infobox}
\begin{lstlisting}
#! /usr/bin/perl -w

open (FILE, 'input.txt') or die "$!";
while (<FILE>) {
	if (m/text file/) {
		for (1..10) {
			<FILE>;
		}
		print;
		last;
	}
}
close FILE;
\end{lstlisting}
%
% 2. Counting lines, words and characters in a file
%
\section{Counting lines, words and characters in a file}
\begin{infobox}
\info{\MakeUppercase{Special instructions for this program}}
Use the same input file as in the previous program.
\end{infobox}
\begin{lstlisting}
#!/usr/bin/perl -w
open(FILE, "textfile.txt") or die "Could not open file: $!";

my ($lines, $words, $chars) = (0,0,0);

while (<FILE>) {
    $linecount++;
    $charcount += length($_);
    $wordcount += scalar(split(/\s+/, $_));
}

print ("Total number of lines: $linecount \n");
print ("Total number of words: $wordcount \n");
print ("Total number of characters: $charcount \n");
\end{lstlisting}
%
% 3. Sorting strings
%
\section{Sorting strings}
\begin{lstlisting}
#!/usr/bin/perl -w
my @strings = qw(quantum relativistic classical);
my @sorted = sort @strings;
print "\nSorted strings:\n";
print join "\n", @sorted;
\end{lstlisting}
%
% 4. Checking for prime numbers
%
\vspace{0.5cm}
\section{Checking for prime numbers}
\begin{lstlisting}
#! /usr/bin/perl -w
print "Enter a number:\t";
$number = <>;
$divisor = 0;
$flag = 0;
if ($number != 2) {
	for ($divisor = 2; $divisor < $number ; $divisor++) {
		if ($number % $divisor == 0) {
			$flag = 1;
			last;
		}
	}
} else {
	$flag = 1;
}
if ($flag != 1) {
	print "The number is prime.\n";
} else {
	print "The number is composite.\n";
}
\end{lstlisting}
%
% 5. Finding the roots of a quadratic equation
%
\vspace{0.5cm}
\section{Finding the roots of a quadratic equation}
\begin{lstlisting}
#! /usr/bin/perl -w
use Math::Complex;
INPUT:
print "\nEnter the three non-zero co-efficients of ax^2 + bx + c:\n";
chomp($a=<>, $b=<>, $c=<>);
if ($a == 0) {
	print "The co-efficient a cannot be zero. Try again.\n";
	goto INPUT;
}
($x1, $x2) = solveQuad($a,$b,$c);
print "Root 1 = $x1, Root 2 = $x2\n\n";
sub solveQuad {
	my ($a, $b, $c) = @_;
	my $root = sqrt($b**2 - 4*$a*$c);
	return (-$b + $root)/(2*$a), (-$b-$root)/(2*$a);
}
\end{lstlisting}
\pagebreak
%
% 6. Least square fitting data from a file
%
\section{Least square fitting data from a file}
\begin{infobox}
\info{\MakeUppercase{Special instructions for this program}}
Create a tab-separated input file with two columns (x and y) of numbers.
\end{infobox}
\begin{lstlisting}
#! /usr/bin/perl -w

print "\nEnter the name of the input file:\t";
$file = <>;
print "Enter the number of rows of data:\t";
$rows = <>;

open (INPUT,$file) or die "Could not open the file";
# @line;
@arrayX = ();
@arrayY = ();
$counter = 0;

($i, $sumx, $sumy, $sumxy, $sumx2) = (0, 0, 0, 0, 0);

while(<INPUT>){
    @line = split(/\t/,$_);
    $arrayX[$counter]=$line[0];
    $arrayY[$counter]=$line[1];
    $counter++;
}

for($i=0;$i<$rows;$i++)
    {
        $sumx = $sumx + $arrayX[$i];
        $sumx2 = $sumx2 + $arrayX[$i]*$arrayX[$i];
        $sumy = $sumy + $arrayY[$i];
        $sumxy = $sumxy + $arrayX[$i]*$arrayY[$i];
        
    }
    
$c = (($sumx2*$sumy - $sumx*$sumxy)*1.0/($rows*$sumx2-$sumx*$sumx)*1.0);
$m = (($rows*$sumxy-$sumx*$sumy)*1.0/($rows*$sumx2-$sumx*$sumx)*1.0);
print "\n\nThe line of best fit is y=$m x + $c \n\n";
\end{lstlisting}
\end{document}